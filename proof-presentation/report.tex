\documentclass{article}
\usepackage[utf8]{inputenc}

\title{MATH2710 — Proof Presentation}
\author{Mike Medved}
\date{April 12th, 2023}

\usepackage{color}
\usepackage{amsthm}
\usepackage{amssymb} 
\usepackage{amsmath}
\usepackage{lmodern}
\usepackage{mathtools, nccmath}
\usepackage{listings}
\usepackage[margin=1in]{geometry} 
\usepackage[table,xdraw,dvipsnames]{xcolor}
\usepackage{tikz}
\usepackage{pgfplots}

\usepackage{xparse}
%
\DeclarePairedDelimiterX{\set}[1]{\{}{\}}{\setargs{#1}}
\NewDocumentCommand{\setargs}{>{\SplitArgument{1}{;}}m}
{\setargsaux#1}
\NewDocumentCommand{\setargsaux}{mm}
{\IfNoValueTF{#2}{#1} {#1\,\delimsize|\,\mathopen{}#2}}%{#1\:;\:#2}

\parindent = 0pt

\newtheorem*{thm}{Theorem}

\begin{document}

\maketitle

\section*{Introduction}

In this report, we will prove that the sum of two independent normal random variables is also a normal random variable, such that:

\begin{align*}
    X \sim \mathcal{N}(\mu_X, \sigma^2_X) \hspace{0.5cm} Y \sim \mathcal{N}(\mu_Y, \sigma^2_Y) \\
    X + Y = Z \sim \mathcal{N}(\mu_X + \mu_Y, \sigma^2_X + \sigma^2_Y)
\end{align*}

\section*{Preliminaries}

\subsection*{The Moment Generating Function}

Let $X$ be a random variable with distribution function $F_X(x)$. The moment generating function of $X$ is defined as follows:

\begin{equation*}
    \phi_X(t) = \mathbb{E} \left[ e^{tX} \right]
\end{equation*}

$\hfill \break$
\textbf{Property:} When added, the moment generating functions of two independent random variables $X$ and $Y$ is equal to the product of their individual moment generating functions:

\begin{align*}
    \phi_{X + Y}(t) &= \phi_X(t) \cdot \phi_Y(t) \\ 
\end{align*}

Since $X$ and $Y$ are independent with respect to one-another, we have the right to multiply their moment generating functions.

\begin{align*}
    &\mathbb{E} \left[ e^{tX} \right] \cdot \mathbb{E} \left[ e^{tY} \right] \\
    &= \mathbb{E} \left(e^{t(X+Y)}\right)
\end{align*}

\newpage
\section*{Proof}

\begin{thm}
    Let $X$ and $Y$ be independent normal random variables with mean $\mu$ and variance $\sigma^2$. Prove that the random variable $X + Y = Z$ is a normal random variable $Z \sim \mathcal{N}(\mu_X + \mu_Y, \sigma^2_X + \sigma^2_Y)$.
\end{thm}

\begin{proof}
    Let $X, Y$ be two independent normal random variables such that:
    
    \begin{align*}
        X \sim \mathcal{N}(\mu_X, \sigma^2_X) \hspace{0.5cm} Y \sim \mathcal{N}(\mu_Y, \sigma^2_Y)
    \end{align*}
    
    We aim to prove that the random variable $Z = X + Y$ is also a normal random variable with mean $\mu_X + \mu_Y$ and variance $\sigma^2_X + \sigma^2_Y$.
    
    $\hfill \break$
    In this case, we know that $X, Y$ are both independent of one another, and are normal random variables. Therefore, we can use the sum of their moment generating functions to prove that $Z$ is also a normal random variable.

    $\hfill \break$
    Let $t \in \mathbb{R}$, then the moment generating function of $X \sim \mathcal{N}(\mu_X, \sigma^2_X)$ is the following:

    \begin{equation*}
        \phi_X(t) = \mathbb{E} \left[ e^{tX} \right] = \text{exp}\left({t\mu_X - \frac{\sigma_X^2 t^2}{2}}\right)
    \end{equation*}

    Similarly, the moment generating function of $Y \sim \mathcal{N}(\mu_Y, \sigma^2_Y)$ is the following:

    \begin{equation*}
        \phi_Y(t) = \mathbb{E} \left[ e^{tY} \right] = \text{exp}\left({t\mu_Y - \frac{\sigma_Y^2 t^2}{2}}\right)
    \end{equation*}

    Therefore, the moment generating function of $X + Y$ will be:

    \begin{align*}
        \phi_{X+Y}(t) &= \phi_{X}(t) \cdot \phi_{Y}(t) \\
        &= \text{exp}\left({t\mu_X - \frac{\sigma_X^2 t^2}{2}}\right) \cdot \text{exp}\left({t\mu_Y - \frac{\sigma^2_Y t^2}{2}}\right) \\
        &= \text{exp}\left({t\left(\mu_X+\mu_Y\right)-\frac{\left(\sigma^2_X + \sigma^2_Y\right) \cdot t^2}{2}}\right) \\
    \end{align*}

    This is exactly the moment generating of a normal random variable $\mathcal{N}(\mu_X + \mu_Y, \sigma^2_X + \sigma^2_Y)$, which is what we aimed to prove.
    
    $\hfill \break$
    Hence, $X+Y$ is a normal random variable $Z \sim \mathcal{N}(\mu_X + \mu_Y, \sigma^2_X + \sigma^2_Y)$.
\end{proof}

\end{document}